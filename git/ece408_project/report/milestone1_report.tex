\documentclass[11pt]{article}

%\usepackage{pstricks,pst-node}
\usepackage{geometry}
 \geometry{
 a4paper,
 total={170mm,257mm},
 left=20mm,
 top=20mm,
 }
\usepackage{alltt,fullpage,graphics,color,epsfig,amsmath, amssymb,color,verbatim}
\usepackage{hyperref}
\usepackage{boxedminipage}
\usepackage{listings}
\usepackage{xcolor}
%\usepackage[skip=2pt]{caption}
\lstset { %
    language=C++,
    %backgroundcolor=\color{black!5}, % set backgroundcolor
    basicstyle=\footnotesize,% basic font setting
}

%\newcommand{\edgee}[1]{\begin{math}\stackrel{#1}{\longrightarrow}\end{math}}

\newcommand{\floor}[1]{\lfloor #1 \rfloor}
\newcommand{\ceil}[1]{\lceil #1 \rceil}
\setlength\parindent{0pt}

\begin{document}
\setlength{\parskip}{.1 in}

\begin{center}
\LARGE
\textbf{ECE 408 Applied Parallel Programming, Milestone 1}
\\[0.5ex]
\normalsize
\textbf{deaplearners}\\
\textbf{Yiliang Wang (wang513), Shuchen Zhang (szhan114), Tao Sun (taosun2)}
\end{center}
%----------------------------------------------------------------------
\textbf{Q1} Include a list of all kernels that collectively consume more than 90\% of the program time.
\begin{table}[!htb]
\centering
\begin{tabular}{|c|c|c|}
\hline
\textbf{Kernel}                               & \textbf{Time} & \textbf{Cummulative Time} \\ \hline
CUDA memcpy HtoD                              & 37.73\%       & 37.73\%                   \\ \hline
volta\_scudnn\_128x32\_relu\_interior\_nn\_v1 & 21.47\%       & 59.20\%                   \\ \hline
cudnn::detail::implicit\_convolve\_sgemm      & 21.32\%       & 80.52\%                   \\ \hline
cudnn::detail::activation\_fw\_4d\_kernel     & 7.45\%        & 87.97\%                   \\ \hline
volta\_sgemm\_128x128\_tn                     & 6.87\%        & 94.84\%                   \\ \hline
\end{tabular}
\end{table}

\textbf{Q2} Include a list of all CUDA API calls that collectively consume more than 90\% of the program time.
\begin{table}[!htb]
\centering
\begin{tabular}{|c|c|c|}
\hline
\textbf{Kernel}           & \textbf{Time} & \textbf{Cummulative Time} \\ \hline
cudaStreamCreateWithFlags & 38.74\%       & 38.74\%                   \\ \hline
cudaMemGetInfo            & 36.72\%       & 75.46\%                   \\ \hline
cudaFree                  & 21.46\%       & 96.92                     \\ \hline
\end{tabular}
\end{table}

\textbf{Q3} Include an explanation of the difference between kernels and API calls.\\
Kernels are user-defined C functions. When called, kernels are executed multiple times in parallel by a number of different CUDA threads, as opposed to regular C functions executing only once.

The API calls are an interface provided by CUDA to facilitate users familiar with the C programming language to easily write GPU programs. The API calls provide implicit initialization, context management, and module management to ease the device code management.

\textbf{Q4} Show output of rai running MXNet on the CPU.
\begin{lstlisting}
*Running /usr/bin/time python m1.1.py
Loading fashion-mnist data... done
Loading model... done
New Inference
EvalMetric: {'accuracy': 0.8177}
19.83user 3.97system 0:13.42elapsed 177%CPU (0avgtext+0avgdata 5955540maxresident)k
0inputs+2856out
puts (0major+1586517minor)pagefaults 0swaps
\end{lstlisting}

\textbf{Q5} List program run time on the CPU.
The program run time on the CPU is 13.42 seconds.
\begin{lstlisting}
19.83user 3.97system 0:13.42elapsed 177%CPU (0avgtext+0avgdata 5955540maxresident)k
0inputs+2856out
puts (0major+1586517minor)pagefaults 0swaps
\end{lstlisting}

\textbf{Q6} Show output of rai running MXNet on the GPU.
\begin{lstlisting}
Running /usr/bin/time python m1.2.py
Loading fashion-mnist data... done
Loading model... done
New Inference
EvalMetric: {'accuracy': 0.8177}
\end{lstlisting}
 
\textbf{Q7} List program run time on the GPU.
The program run time on the CPU is 4.59 seconds.
\begin{lstlisting}
4.37user 2.67system 0:04.59elapsed 153%CPU (0avgtext+0avgdata 2833892maxresident)k
0inputs+4568outputs (0major+703015minor)pagefaults 0swaps
\end{lstlisting}
%------------------------------------------------------------------------
\end{document}

